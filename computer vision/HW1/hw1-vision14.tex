\documentclass[10pt]{article}

\usepackage{amsmath,amscd}
\usepackage{amssymb,array}
\usepackage{amsfonts,latexsym}
\usepackage{graphicx,subfig,wrapfig}
\usepackage{times}
\usepackage{psfrag,epsfig}
\usepackage{verbatim}
\usepackage{tabularx}
\usepackage[pagebackref=true,breaklinks=true,letterpaper=true,colorlinks,bookmarks=false]{hyperref}

\DeclareMathOperator*{\rank}{rank}
\DeclareMathOperator*{\trace}{trace}
\DeclareMathOperator*{\range}{range}
\DeclareMathOperator*{\spn}{span}
\DeclareMathOperator*{\acos}{acos}
\DeclareMathOperator*{\argmax}{argmax}

\newcommand{\matlab}[1]{\texttt{#1}}
\newcommand{\setname}[1]{\textsl{#1}}
\newcommand{\Ce}{\mathbb{C}}

\newenvironment{mfunction}[1]{
\noindent
\tabularx{\linewidth}{>{\ttfamily}rX}
\hline
\multicolumn{2}{l}{\textbf{Function \matlab{#1}}}\\
\hline
}{\\\endtabularx}

\newcommand{\parameters}{\multicolumn{2}{l}{\textbf{Parameters}}\\}

\newcommand{\fdescription}[1]{\multicolumn{2}{p{0.96\linewidth}}{

\textbf{Description}

 #1}\\\hline}

\newcommand{\retvalues}{\multicolumn{2}{l}{\textbf{Returned values}}\\}
\def\0{\boldsymbol{0}}
\def\b{\boldsymbol{b}}
\def\bmu{\boldsymbol{\mu}}
\def\e{\boldsymbol{e}}
\def\u{\boldsymbol{u}}
\def\x{\boldsymbol{x}}
\def\v{\boldsymbol{v}}
\def\w{\boldsymbol{w}}
\def\N{\boldsymbol{N}}
\def\X{\boldsymbol{X}}
\def\Y{\boldsymbol{Y}}
\def\A{\boldsymbol{A}}
\def\B{\boldsymbol{B}}
\def\y{\boldsymbol{y}}
\def\cX{\mathcal{X}}
\def\transpose{\top} 

%\long\def\answer#1{{\bf ANSWER:} #1}
\long\def\answer#1{}
\newcommand{\myhat}{\widehat}
\long\def\comment#1{}
\newcommand{\eg}{{e.g.,~}}
\newcommand{\ea}{{et al.~}}
\newcommand{\ie}{{i.e.,~}}

\newcommand{\db}{{\boldsymbol{d}}}
\renewcommand{\Re}{{\mathbb{R}}}
\newcommand{\Pe}{{\mathbb{P}}}

\hyphenation{MATLAB}
\usepackage[margin=1in]{geometry}


\begin{document}

\title{
\vspace{-19mm}
Computer Vision (600.461/600.661)\\
Homework 1: Mathematical Background}
\author{Instructor: Ren\'e Vidal}
\date{Due 09/11/2014, 11.59PM Eastern}

\maketitle

\begin{enumerate}

\item \textbf{Properties of Symmetric Matrices.} 
Let $S\in\Re^{n\times n}$ be a real symmetric matrix. Show that: 
%
\begin{enumerate}
\item All the eigenvalues of $S$ are real, \ie $\sigma(S) \subset \Re$.
\vspace{5mm}
\newline Assume $\lambda \in \Im $
\newline $S \tilde{v} = \tilde{\lambda} \tilde{v} $
\newline $
\item Let $(\lambda,v)$ be an eigenvalue-eigenvector pair. If $\lambda_i \ne \lambda_j$, then $v_i \perp v_j$; \ie  eigenvectors corresponding to distinct eigenvalues are orthogonal.
\item There always exist $n$ orthonormal eigenvectors of $S$, which form a basis of $\Re^n$.
\item $S$ is positive definite (positive semidefinite) if and only if all of its eigenvalues are positive (non-negative), \ie  $S \succ 0$ ($S \succeq 0$), iff $ \forall i =
1,2,\ldots,n$, $\lambda_i > 0$ ($\lambda_i \ge 0$).
\item If $\lambda_1 \ge \lambda_2 \ge \cdots \ge \lambda_n $ are the sorted eigenvalues of $S$, then $\max\limits_{\|\x\|_2 = 1} \x^\transpose S \x = \lambda_1$ and
$\min\limits_{\|\x\|_2 = 1} \x^\transpose S \x  = \lambda_n$.
\end{enumerate}

\item \textbf{Properties of the SVD.}
Let $ A = U \Sigma V^\transpose$ be the SVD of a matrix $A \in \Re^{m\times n}$ of rank $r$. Show that:
%
\begin{enumerate}
\item $A \v_j = \sigma_j \u_j$ for $j=1,\dots,r$ and $A^\transpose \u_j = \sigma \v_j$ for $j=1,\dots,r$.
\item The range or image of $A$ is spanned by the left singular vectors of $A$ associated with its nonzero singular values, \ie $\range(A) = \spn\{\u_i\}_{i=1}^r$.
\item The kernel or null space of $A$ is spanned by the right singular vectors of $A$ associated with its zero singular values, \ie $\ker(A) = \spn\{\v_i\}_{i=r+1}^m$.
\item The squared Frobenius norm of $A$ is equal to the sum of the squared singular values of $A$, \ie $\|A\|_F^2 = \sum_{ij} a_{ij}^2 = \sum_{k=1}^r \sigma_k^2$.
\item The right singular vector of $A$ associated to its smallest singular value, $\v_m$,  is a solution to the optimization problem $\min\limits_{\x} \|A\x\|_2^2$ such that $\|\x\|_2 = 1$.
\end{enumerate}

\item \textbf{Least Squares.}
Recall that the pseudo inverse of a matrix $A\in\Re^{m\times n}$ is the unique matrix $A^\dag \in \Re^{n\times m}$ such that: (i) $AA^\dag A = A$, (ii) $A^\dag A A^\dag = A^\dag$, (iii) $(AA^\dag )^\transpose = AA^\dag$, and (iv) $(A^\dag A)^\transpose = A^\dag A$. Let $ A = U \Sigma V^\transpose$ be the SVD of $A$, let $r = \rank(A)$ and let $\b\in\Re^m$. 
Show that:

\begin{enumerate}
\item The pseudo-inverse of $A$ is given by $A^\dag = V_r \Sigma_r^{-1} U_r^\transpose$, where $A = U_r \Sigma_r V_r ^ \transpose$ is the compact SVD of $A$. 

\item $\x^* = A^\dag \b $ is a solution to the optimization problem $\min\limits_{\x} \| A \x - \b \|_2^2$. When is $\x^*$ the unique solution?

\item If $\b \in \range (A)$, $\x^* = A^\dag \b $ is the solution to the optimization problem $\min\limits_{\x} \| \x \|_2^2$ such that $A \x = \b$. 
\end{enumerate}



\end{enumerate}

\noindent\textbf{Submission instructions.} Send email to \href{mailto:vision14jhu@gmail.com?subject=600.692:HW1}{vision14jhu@gmail.com} with subject  \href{mailto:vision14jhu@gmail.com?subject=600.461/600.661:HW1}{600.461/600.661:HW1} and attachment \matlab{firstname-lastname-hw1-vision14.zip} or \matlab{firstname-lastname-hw1-vision14.tar.gz}. The attachment should have the following content:
%
\begin{enumerate}
\item A file called \matlab{hw1.pdf} containing your answers to each one of the analytical questions. If at all possible,
you should generate this file using the latex template
\href{http://www.vision.jhu.edu/teaching/vision/vision14/Homeworks/hw1-learning14.tex}{hw1-vision14.tex}.
If not possible, you may use another editor, or scan your handwritten solutions.
But note that you must submit a single PDF file with all your answers.

%\item For coding questions, submit a file called \matlab{README}, 
%which contains instructions on how to run your code. Use
%separate directories for each coding problem. Each directory should
%contain all the functions and scripts you are asked to write in separate
%files. For example, for HW1 the structure of what you should submit could look like
%\begin{enumerate}
%\item \matlab{README}
%\item \matlab{hw1.pdf}
%\item
%\matlab{hw1q3}: \matlab{hw1q3c.m}, \matlab{hw1q3e.m}
%\item \matlab{hw1q4}: \matlab{hw1q4b.m}, \matlab{hw1q4c.m}
%\end{enumerate}
%
%The TA will run your scripts to generate the results. Thus, your
%script should include all needed plotting commands so that figures
%pop up automatically. Please make sure that the figure numbers match
%those you describe in \matlab{hw1.pdf}. You do not
%need to submit input or output images. The output images should be
%automatically generated by your scripts so that the TA can see the results
%by just running the scripts. In writing your code, you should assume that
%the TA will place the input images in the directory that is relevant to the
%question solved by your script. Also, make sure to comment your code properly.

\end{enumerate}

\end{document}
