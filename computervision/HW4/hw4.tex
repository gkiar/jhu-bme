\documentclass[10pt]{article}

\usepackage{amsmath,amscd}
\usepackage{amssymb,array}
\usepackage{amsfonts,latexsym}
\usepackage{graphicx,subfig,wrapfig}
\usepackage{times}
\usepackage{psfrag,epsfig}
\usepackage{verbatim}
\usepackage{tabularx}
\usepackage[pagebackref=true,breaklinks=true,letterpaper=true,colorlinks,bookmarks=false]{hyperref}

\DeclareMathOperator*{\rank}{rank}
\DeclareMathOperator*{\trace}{trace}
\DeclareMathOperator*{\range}{range}
\DeclareMathOperator*{\spn}{span}
\DeclareMathOperator*{\acos}{acos}
\DeclareMathOperator*{\argmax}{argmax}

\newcommand{\matlab}[1]{\texttt{#1}}
\newcommand{\setname}[1]{\textsl{#1}}
\newcommand{\Ce}{\mathbb{C}}

\newenvironment{mfunction}[1]{
\noindent
\tabularx{\linewidth}{>{\ttfamily}rX}
\hline
\multicolumn{2}{l}{\textbf{Function \matlab{#1}}}\\
\hline
}{\\\endtabularx}

\newcommand{\parameters}{\multicolumn{2}{l}{\textbf{Parameters}}\\}

\newcommand{\fdescription}[1]{\multicolumn{2}{p{0.96\linewidth}}{

\textbf{Description}

 #1}\\\hline}

\newcommand{\retvalues}{\multicolumn{2}{l}{\textbf{Returned values}}\\}
\def\0{\boldsymbol{0}}
\def\b{\boldsymbol{b}}
\def\bmu{\boldsymbol{\mu}}
\def\e{\boldsymbol{e}}
\def\u{\boldsymbol{u}}
\def\x{\boldsymbol{x}}
\def\v{\boldsymbol{v}}
\def\w{\boldsymbol{w}}
\def\N{\boldsymbol{N}}
\def\X{\boldsymbol{X}}
\def\Y{\boldsymbol{Y}}
\def\A{\boldsymbol{A}}
\def\B{\boldsymbol{B}}
\def\y{\boldsymbol{y}}
\def\cX{\mathcal{X}}
\def\transpose{\top} 

%\long\def\answer#1{{\bf ANSWER:} #1}
\long\def\answer#1{}
\newcommand{\myhat}{\widehat}
\long\def\comment#1{}
\newcommand{\eg}{{e.g.,~}}
\newcommand{\ea}{{et al.~}}
\newcommand{\ie}{{i.e.,~}}

\newcommand{\db}{{\boldsymbol{d}}}
\renewcommand{\Re}{{\mathbb{R}}}
\newcommand{\Pe}{{\mathbb{P}}}

\hyphenation{MATLAB}
\usepackage[margin=1in]{geometry}


\begin{document}

\title{
\vspace{-19mm}
Computer Vision (600.461/600.661)\\
Homework 4: Feature Matching and Optical Flow}
\author{Greg Kiar}


\maketitle

\begin{enumerate}

\item \textbf{(15 Points) Corner localization via quadratic fit.} 
\begin{align*}
E(Q,b,c) &= \sum_{u=1}w(x+u) \begin{bmatrix} \frac{1}{2}(x+u)^TQ(x+u) + b^T(x+u) + c - r(x+u) \end{bmatrix}^2
\end{align*}
From here, let $x_i = x+u$.
\begin{align*}
\underset{Q,b,c}{\operatorname{min}} E(Q,b,c) &=
\underset{Q,b,c}{\operatorname{min}} \sum_{i=1}w_i \begin{bmatrix} \frac{1}{2}x_i^TQx_i + b^Tx_i + c - r_i \end{bmatrix}^2 \\
\frac{\partial}{\partial c} E &= \frac{\partial}{\partial c} \sum_{i=1} w_i \begin{bmatrix} \frac{1}{2} x_i^TQX_i + b^Tx_i + c - r_i \end{bmatrix}^2 \\
0 &= 2  \sum_{i=1} w_i \begin{bmatrix} \frac{1}{2} x_i^TQX_i + b^Tx_i - r_i \end{bmatrix} + 2 \sum_{i=1} w_ic \\
c \sum_{i=1}w_i &= -\sum_{i=1} w_i \begin{bmatrix} \frac{1}{2} x_i^TQX_i + b^Tx_i - r_i \end{bmatrix} \\
c^* & = -\frac{\sum_{i=1} w_i \begin{bmatrix} \frac{1}{2} x_i^TQX_i + b^Tx_i - r_i \end{bmatrix}}{\sum_{i=1}w_i} \\
c^* & = -w \begin{pmatrix} \frac{1}{2} \bar{x}^TQ\bar{x} - b^T\bar{x} + \bar{r} \end{pmatrix}
\end{align*}
Now, subbing $c^*$ into the error equation we get,
\begin{align*}
\underset{Q,b,c^*}{\operatorname{min}} E(Q,b,c^*) &=
\underset{Q,b,c^*}{\operatorname{min}} \sum_{i=1}w_i \begin{bmatrix} \frac{1}{2}x_i^TQx_i + b^Tx_i + c^* - r_i \end{bmatrix}^2 \\
 &= \underset{Q,b,c^*}{\operatorname{min}} \sum_{i=1}w_i \begin{bmatrix} \frac{1}{2}x_i^TQx_i + b^Tx_i - \frac{1}{2} \bar{x}^TQ\bar{x} - b^T\bar{x} + \bar{r} - r_i \end{bmatrix}^2 \\
&= \underset{Q,b,c^*}{\operatorname{min}} \sum_{i=1}w_i \begin{bmatrix} \begin{pmatrix} \frac{1}{2}x_i^TQx_i - \frac{1}{2} \bar{x}^TQ\bar{x} \end{pmatrix} + \begin{pmatrix} b^Tx_i -  b^T\bar{x} \end{pmatrix} - \begin{pmatrix} r_i - \bar{r} \end{pmatrix} \end{bmatrix}^2 \\
&= \underset{Q,b,c^*}{\operatorname{min}} w \begin{bmatrix} \frac{1}{2}\tilde{x}^TQ\tilde{x} + b^T\tilde{x} - \tilde{r} \end{bmatrix}^2 \\
\frac{\partial}{\partial b} E &= \frac{\partial}{\partial b}  w \begin{bmatrix} \frac{1}{2}\tilde{x}^TQ\tilde{x} + b^T\tilde{x} - \tilde{r} \end{bmatrix}^2 \\
0 &= 2 w \begin{bmatrix} \frac{1}{2}\tilde{x}^TQ\tilde{x} + b^T\tilde{x} - \tilde{r} \end{bmatrix} \tilde{x}^T \\
0 &= 2w \begin{bmatrix} \frac{1}{2}\tilde{x}^TQ\tilde{x}\tilde{x}^T + b^T\tilde{x}\tilde{x}^T - \tilde{r}\tilde{x}^T \end{bmatrix} \\
b^* &= \begin{pmatrix} \frac{1}{2}\tilde{x}^TQ\tilde{x}\tilde{x}^T - \tilde{r}\tilde{x}^T \end{pmatrix}\begin{pmatrix} \tilde{x}\tilde{x}^T \end{pmatrix}^{-1} \\
\end{align*}
Now, subbing $b^*$ into the same equation, this time isolating for our last variable, $Q$,
\begin{align*}
\underset{Q,b^*,c^*}{\operatorname{min}} E(Q,b^*,c^*) &=
\underset{Q,b^*,c^*}{\operatorname{min}} w \begin{bmatrix} \frac{1}{2}\tilde{x}^TQ\tilde{x} +\begin{pmatrix}\begin{pmatrix} \frac{1}{2}\tilde{x}^TQ\tilde{x}\tilde{x}^T - \tilde{r}\tilde{x}^T \end{pmatrix}\begin{pmatrix} \tilde{x}\tilde{x}^T \end{pmatrix}^{-1} \end{pmatrix}^T\tilde{x} - \tilde{r} \end{bmatrix}^2 \\
\end{align*}
From this form we can see that the solution will be in the form of least squares for $Q$. Since it is a maximum, $Q$ must be negative semi-definite. Therefore we can constrain the solution to be only negative eigen values, $\lambda$.

\item \textbf{(20 Points) Feature point matching under a 2D rigid body motion.}\\
\begin{align*}
E(R,t) &= \sum_{j=1}\lVert y_j - Rx_j - t \rVert_2^2
\end{align*}
We know that,from class,
\begin{align*}
\underset{R,t}{\operatorname{min}} E(R,t)&:\\
t^* &= \bar{Y} - R\bar{X} \\
R^* &= \underset{R}{\operatorname{min}}\lVert Y-RX \rVert_F^2\\
&=  \underset{R}{\operatorname{min}}\lVert Y \rVert_F^2 - 2\langle Y, RX \rangle + \lVert RX \rVert_F^2\\
&= \underset{R}{\operatorname{min}} -\langle Y,RX \rangle \\
&= \underset{R}{\operatorname{max}} \langle Y,RX \rangle \\
R^* &= \underset{R}{\operatorname{max}} \text{ trace}(Y^T RX) 
\end{align*}
We also know from class that $R$ is of the form:
\begin{align*}
R &=
\begin{pmatrix}
\cos\theta & -\sin\theta \\
\sin\theta & \cos\theta\\
\end{pmatrix}\\
R &= I\cos\theta + 
\begin{pmatrix}
0 & -1 \\ 1 & 0
\end{pmatrix}
\sin\theta \\
\therefore R^* &=
\begin{pmatrix}
\cos\theta^* & -\sin\theta^* \\
\sin\theta^* & \cos\theta^*\\
\end{pmatrix}
\end{align*}
Where, 
\begin{align*}
\theta^* &= \underset{\theta}{\operatorname{max}} \text{ trace}(Y^YRX)^T\\
&= \underset{\theta}{\operatorname{max}} \text{ trace}(X^T R^TY)\\
&= \underset{\theta}{\operatorname{max}} \text{ trace}(X^T(I\cos\theta + 
\begin{pmatrix}
0 & 1 \\ -1 & 0
\end{pmatrix}
\sin\theta)Y)\\
&= \underset{\theta}{\operatorname{max}} \text{ trace}(X^T Y\cos\theta) + \text{ trace}(X^T 
\begin{pmatrix}
0 & 1 \\ -1 & 0
\end{pmatrix}
Y\sin\theta)\\
0 &= \frac{\partial}{\partial\theta}
\begin{bmatrix}
\text{ trace}(X^T Y\cos\theta) + \text{ trace}(X^T 
\begin{pmatrix}
0 & 1 \\ -1 & 0
\end{pmatrix}
Y\sin\theta)
\end{bmatrix}\\
\text{ trace}(X^T Y)\sin\theta &= \text{ trace}(X^T 
\begin{pmatrix}
0 & 1 \\ -1 & 0
\end{pmatrix}
Y)\cos\theta \\
\end{align*}
We can now see that
\begin{align*}
\theta^* &= \arctan
\begin{pmatrix}
\frac{\text{ trace}(X^T
\begin{pmatrix}
0 & 1 \\ -1 & 0
\end{pmatrix}
Y)}{\text{ trace}(X^TY)}
\end{pmatrix} \\
\sin(\theta^*) &= \frac{\text{ trace}(X^T
\begin{pmatrix}
0 & 1 \\ -1 & 0
\end{pmatrix}
Y)}{\sqrt{\text{ trace}(X^TY)^2 + \text{ trace}(X^T
\begin{pmatrix}
0 & 1 \\ -1 & 0
\end{pmatrix}
Y)^2}} \\
\cos(\theta^*) &= \frac{\text{ trace}(X^TY)}{\sqrt{\text{ trace}(X^TY)^2 + \text{ trace}(X^T
\begin{pmatrix}
0 & 1 \\ -1 & 0
\end{pmatrix}
Y)^2}} \\
\end{align*}
$\therefore$ we can sub in $\sin(\theta^*)$ and $\cos(\theta^*)$ into $R$ and find that\\
\begin{align*}
R &= \frac{
\begin{pmatrix}
\text{ trace}(X^T Y) & -\text{ trace}(X^T \begin{pmatrix} 0 & 1 \\ -1 & 0 \end{pmatrix} Y) \\
\text{ trace}(X^T \begin{pmatrix} 0 & 1 \\ -1 & 0 \end{pmatrix} Y)  & \text{ trace}(X^T Y) \\
\end{pmatrix}}{\sqrt{\text{ trace}(X^TY)^2 + \text{ trace}(X^T \begin{pmatrix} 0 & 1 \\ -1 & 0 \end{pmatrix} Y)^2}}
\end{align*}


\item \textbf{(15 Points) Optical flow with changes in illumination.}
\begin{align*}
I(x_u, y+v, t+ 1) &= a I(x,y,t) + b
\end{align*}
Notice that the above matches the form
\begin{align*}
J(x,y,t) &= a I(x,y,t) + b
\end{align*}
Recognizing the above as a standard affine transform we can conclude that
\begin{align*}
b &= \bar{J} - a \bar{I}\\
a &= \tilde{J}\tilde{I}^T(\tilde{I}\tilde{I}^T)^{-1}
\end{align*}
Where,
\begin{align*}
J &= I(x+u, y+v, t+1)\\
\tilde{J} &= J - \bar{J}\\
\tilde{I} &= I - \bar{I}
\end{align*}
Now, applying the brightness constancy contraint (BCC) and making the assumption of small motion,
\begin{align*}
I(x+u,y+v,t+1) &= aI(x, y, t) +b + uI_x + vI_y \\
0 &= aI(x,y,t) + b - I(x+u, y+v,t+1) + uI_x + vI_y \\
0 &= (a-1)I(x, y, t) + b (I(x, y, t) - I(x+u,y+v,t+1)) + uI_x + vI_y \\
0 &= (a-1)I(x, y, t) + b + I_t + uI_x + vI_y\\
0 &= (a-1)I(x, y, t) + b + I_t + \nabla I \begin{bmatrix} u & v \end{bmatrix} \\
\end{align*}
From here, we have 1 equation and 2 unknowns, $u$ and $v$. We now make an assumption that nearby points have a constant optical flow, $u$ and $v$, and solve for multiple points at the same time. We will choose a $5$x$5$ neighborhood around our point of interest.
\begin{align*}
0 &= (a-1)I(p_i) + b + I_t(p_i) + \nabla I(p_i)\begin{bmatrix}u & v\end{bmatrix} \\
\begin{bmatrix} I_x(p_1) & I_y(p_1) \\ I_x(p_2) & I_y(p_2) \\ \vdots & \vdots \\ I_x(p_{25}) & I_y(p_{25}) \end{bmatrix}
\begin{bmatrix} u \\ v \end{bmatrix} &= - \begin{bmatrix} I_t(p_1) \\ I_t(p_2) \\ \vdots \\ I_t(p_{25}) \end{bmatrix}
- (a-1) \begin{bmatrix} I(p_1) \\ I(p_2) \\ \vdots \\ I(p_{25}) \end{bmatrix} - \begin{bmatrix} b(p_1) \\ b(p_2) \\ \vdots \\ b(p_{25}) \end{bmatrix} \\
\text{Where we notice the above equation takes the form}\\
A x &= b
\end{align*}
This form is commonly seen and can be solved using the least squares method to give the following general and unque solutions, respectively
\begin{align*}
\min \lVert A&x - b \rVert^2 \\
A^TAx &= A^Tb \\
x &= (A^TA)^{-1}A^Tb
\end{align*}
\begin{align*}
\min \begin{Vmatrix} \begin{bmatrix} I_x(p_1) & I_y(p_1) \\ I_x(p_2) & I_y(p_2) \\ \vdots & \vdots \\ I_x(p_{25}) & I_y(p_{25}) \end{bmatrix}
\begin{bmatrix} u \\ v \end{bmatrix} + \begin{pmatrix}\begin{bmatrix} I_t(p_1) \\ I_t(p_2) \\ \vdots \\ I_t(p_{25}) \end{bmatrix}
- (a-1) \begin{bmatrix} I(p_1) \\ I(p_2) \\ \vdots \\ I(p_{25}) \end{bmatrix} - \begin{bmatrix} b(p_1) \\ b(p_2) \\ \vdots \\ b(p_{25}) \end{bmatrix} \end{pmatrix} \end{Vmatrix}^2   \\
\end{align*}
\begin{align*}
\begin{bmatrix}\sum I_x^2 & \sum I_xI_y \\ \sum I_xI_y & \sum I_y^2\end{bmatrix} \begin{bmatrix} u \\ v \end{bmatrix} &= - \begin{bmatrix} \sum I_xQ \\ \sum I_yQ \end{bmatrix}\\
\text{Where, } \\
Q &= \begin{bmatrix} I_t(p_1) \\ I_t(p_2) \\ \vdots \\ I_t(p_{25}) \end{bmatrix}
- (a-1) \begin{bmatrix} I(p_1) \\ I(p_2) \\ \vdots \\ I(p_{25}) \end{bmatrix} - \begin{bmatrix} b(p_1) \\ b(p_2) \\ \vdots \\ b(p_{25}) \end{bmatrix}
\end{align*}

\end{enumerate}
\end{document}
