\documentclass[10pt]{article}

\usepackage{amsmath,amscd}
\usepackage{amssymb,array}
\usepackage{amsfonts,latexsym}
\usepackage{wasysym}
\usepackage{graphicx,subfig,wrapfig}
\usepackage{times}
\usepackage{listings}
\usepackage{psfrag,epsfig}
\usepackage{verbatim}
\usepackage{tabularx}
\usepackage[pagebackref=true,breaklinks=true,letterpaper=true,colorlinks,bookmarks=false]{hyperref}

\DeclareMathOperator*{\rank}{rank}
\DeclareMathOperator*{\trace}{trace}
\DeclareMathOperator*{\range}{range}
\DeclareMathOperator*{\spn}{span}
\DeclareMathOperator*{\acos}{acos}
\DeclareMathOperator*{\argmax}{argmax}

\newcommand{\matlab}[1]{\texttt{#1}}
\newcommand{\setname}[1]{\textsl{#1}}
\newcommand{\Ce}{\mathbb{C}}

\newenvironment{mfunction}[1]{
\noindent
\tabularx{\linewidth}{>{\ttfamily}rX}
\hline
\multicolumn{2}{l}{\textbf{Function \matlab{#1}}}\\
\hline
}{\\\endtabularx}

\newcommand{\parameters}{\multicolumn{2}{l}{\textbf{Parameters}}\\}

\newcommand{\fdescription}[1]{\multicolumn{2}{p{0.96\linewidth}}{

\textbf{Description}

 #1}\\\hline}

\newcommand{\retvalues}{\multicolumn{2}{l}{\textbf{Returned values}}\\}
\def\0{\boldsymbol{0}}
\def\b{\boldsymbol{b}}
\def\bmu{\boldsymbol{\mu}}
\def\e{\boldsymbol{e}}
\def\u{\boldsymbol{u}}
\def\x{\boldsymbol{x}}
\def\v{\boldsymbol{v}}
\def\w{\boldsymbol{w}}
\def\N{\boldsymbol{N}}
\def\X{\boldsymbol{X}}
\def\Y{\boldsymbol{Y}}
\def\A{\boldsymbol{A}}
\def\B{\boldsymbol{B}}
\def\y{\boldsymbol{y}}
\def\cX{\mathcal{X}}
\def\transpose{\top} 

%\long\def\answer#1{{\bf ANSWER:} #1}
\long\def\answer#1{}
\newcommand{\myhat}{\widehat}
\long\def\comment#1{}
\newcommand{\eg}{{e.g.,~}}
\newcommand{\ea}{{et al.~}}
\newcommand{\ie}{{i.e.,~}}

\newcommand{\db}{{\boldsymbol{d}}}
\renewcommand{\Re}{{\mathbb{R}}}
\newcommand{\Pe}{{\mathbb{P}}}

\hyphenation{MATLAB}
\usepackage[margin=1in]{geometry}


\begin{document}

\title{
\vspace{-19mm}
SBE III (580.429)\\
Homework 8: DNA Information Content}
\author{Greg Kiar}
\date{Due 10/03/2014}

\maketitle

\begin{enumerate}

\item  
%
\begin{enumerate}
\item 

\item $\because$ the sequence occurs approximately once in the genome of size $3 \times 10^9$, the probability $p$ of it occurring on a give trial is  $p = \dfrac{1}{3 \times 10^9}$.

\item Description of the performance of different filters for gaussian noise:

\end{enumerate}

\item \textbf{(30 points) Color-based face detection.}
\begin{enumerate}
\item Description of MATLAB functions colormap, hsv2rgb, rgb2gray, rgb2hsv, rgb2ntsc, and rgb2ycbcr:
\begin{enumerate}
\item colormap: this function either returns the current colormap or allows you to set a new colormap based on the way you wish your image to be displayed, whether it's remapping brightness, or colours themselves.
\item hsv2rgb: this function can be either used to convert an hsv image to an rgb image, or an hsv colormap to an rgb colormap.
\item rgb2gray: this function behaves similarly to hsv2rgb, except it either converts or maps from an rgb format to a grayscale image. There happens to be an advanced mode as well for this function I saw in Mathworks documentation that performs this operation on the GPU for better parallel processing.
\item rgb2hsv: this function is the exact reverse function of hsv2rgb. It takes an rgb image or map and produces the hsv equivalent.
\item rgb2ntsc: this function is similar to those above, where it converts from rgb to ntsc (or yiq) color mapping systems.
\item rgb2ycbcr: this function is also similar to those above, where it converts from rgb to ycbcr color system. These functions are all valuable depending on what type of color system you're interested in processing images in.
\end{enumerate}
\item Image visibility:
\begin{enumerate}
\item Red: The red image shows the skin colour fairly simiarly to a grayscale version of the entire image. We can clearly see the faces here
\item Green: The green image shows the face colour intensely as well, and perhaps distinguishes fine differences in the facial colour more than the red image. Also, in this image the face colours are quite distinguished from the other colours.
\item Blue: The blue image is similar to the green image, though the blue face colours blend into the other colours in the image more, such as Ross' shirt.
\item Hue: The hue image actually shows a complete absense of faces. This is interesting as it could perhaps be used as a negative to detect where faces are. Unfortunately, the background is also included in this.
\item Saturation: The saturation has the faces showing very differently from the rest of the image. The intensity of them is quite distinct in this image compared to everything else.
\item Value: The value image actually looks almost identical to the red image mentioned earlier. The features of the image are all easily distinguishable.
\end{enumerate}
\item The method developed to remove every part of the image except for the face using RGB was created by trial and error, as suggested. It did not completely remove all other components, but I think that though my solution was certainly not optimal, there will most certainly be some left over colours if we are only detecting faces using RGB.
\item The method developed to remove every part of the image except for the face using HSV was created by trial and error, as suggested. Much like above, it was not perfect, though it was quite successful by thresholding hue and saturation values. I anticipate this being a better method in the general case.
\item After analyzing the different Friends image with the exact algorithms I developed before, same numbers and all, I realized that my filters no longer worked almost at all. I'm not sure if this was the point: to see that using RGB and HSV colour based algorithms for facial recognition are impractical, or if I just went about solving the challenge in a way that didn't transfer well between images. After playing around with the methods more I did come to the conclusion however that the HSV method was superior to the RGB. I think that this is because the RGB method relies too heavily on the background or remaining portions of the foreground to work effectively. The HSV method, in turn, is able to more reliably observe faces through the saturation value even though the hue value was less useful for this particular image.
\end{enumerate}



\end{enumerate}




\end{document}
